%!TEX root = ../Dissertation.tex

\documentclass[%
pdftex,
oneside,			% Einseitiger Druck.
12pt,				% Schriftgroesse
parskip=half,		% Halbe Zeile Abstand zwischen Absätzen.
%topmargin = 10pt,	% Abstand Seitenrand (Std:1in) zu Kopfzeile [laut log: unused]
headheight = 33pt,	% Höhe der Kopfzeile
%headsep = 30pt,	% Abstand zwischen Kopfzeile und Text Body  [laut log: unused]
headsepline,		% Linie nach Kopfzeile.
footsepline,		% Linie vor Fusszeile.
%footheight = 16pt,	% Höhe der Fusszeile
abstracton,		% Abstract Überschriften
DIV=calc,		% Satzspiegel berechnen
BCOR=8mm,		% Bindekorrektur links: 8mm
headinclude=false,	% Kopfzeile nicht in den Satzspiegel einbeziehen
footinclude=false,	% Fußzeile nicht in den Satzspiegel einbeziehen
listof=totoc,		% Abbildungs-/ Tabellenverzeichnis im Inhaltsverzeichnis darstellen
toc=bibliography,	% Literaturverzeichnis im Inhaltsverzeichnis darstellen
]{scrreprt}	% Koma-Script report-Klasse, fuer laengere Bachelorarbeiten alternativ auch: scrbook

% Einstellungen laden
\usepackage{xstring}

\usepackage{lastpage}
\usepackage{fancyhdr}
\newcommand{\einstellung}[1]{%
	\expandafter\newcommand\csname #1\endcsname{}
	\expandafter\newcommand\csname setze#1\endcsname[1]{\expandafter\renewcommand\csname#1\endcsname{##1}}
}
\newcommand{\langstr}[1]{\einstellung{lang#1}}

% Flag für die Selbstständigkeitserklärung, Default: true
\newif\ifselbsterkl
\selbsterklfalse

% Flag für das Wasserzeichen auf dem Deckblatt, default: false
\newif\ifwatermark
\watermarkfalse

% Flag für roten Vertraulichkeitspunkt, default: false
\newif\ifreddot
\reddotfalse

% Flag für gelben Vertraulichkeitspunkt, default: false
\newif\ifyellowdot
\yellowdotfalse

% Flag für grünen Vertraulichkeitspunkt, default: false
\newif\ifgreendot
\greendotfalse

% Flag für das Unterschriftenblatt, default: false
\newif\ifunterschriftenblatt
\unterschriftenblattfalse

% Flag für Einfügen der Seitenzahl bei Verweis auf Kapitel/Abschnitt, default: false
\newif\ifrefWithPages
\refWithPagesfalse

% Flag für Einfügen der Abstracts in deutsch und englisch, default: false
\newif\ifbothabstracts
\bothabstractsfalse

% Flag für Einfügen des Abkürzungsverzeichnis
\newif\ifabkverz
\abkverzfalse

% Flag für Einfügen des Abbildungsverzeichnisses
\newif\ifabbverz
\abbverzfalse

% Flag für Einfügen des Formelverzeichnisses
\newif\ifformelverz
\formelverzfalse

% Flag für Einfügen des Formelgroessenverzeichnisses
\newif\ifformelgroeverz
\formelgroeverzfalse 

% Flag für Einfügen des Listingsverzeichnisses
\newif\iflistverz
\listverzfalse

% Flag für Einfügen des Tabellenverzeichnisses
\newif\iftableverz
\tableverzfalse

% Flag für Einfügen des Sperrvermerks
\newif\ifsperrvermerk
\sperrvermerkfalse

% Flag für Einfügen des Abstracts
\newif\ifabstract
\abstractfalse

% Flag für Anhang
\newif\ifappendix
\appendixfalse

% Flag für Literaturverzeichnis
\newif\ifliteratur
\literaturfalse

% Flag für Glossar
\newif\ifglossar
\glossarfalse

% Flag für Inhaltsverzeichnis
\newif\ifinhalt
\inhaltfalse

\einstellung{autor}
\einstellung{arbeit}
\einstellung{titel}
\einstellung{zeitraum}
\einstellung{modulbeginn}
\einstellung{modulende}
\einstellung{datumAbgabe}
\einstellung{abgabeort}
\einstellung{dhbw}
\einstellung{studiengang}
\einstellung{abschluss}
\einstellung{martrikelnr}
\einstellung{kurs}
\einstellung{jahrgang}
\einstellung{studienhalbjahr}
\einstellung{firma}
\einstellung{firmenort}
\einstellung{abteilung}
\einstellung{betreuer}
\einstellung{gutachter}
\einstellung{sprache}
\einstellung{schriftart}
\einstellung{kapitelabstand}
\einstellung{spaltenabstand}
\einstellung{zeilenabstand}
\einstellung{zitierstil}
\einstellung{selbsterkl} % verfügbare Einstellungen
%%%%%%%%%%%%%%%%%%%%%%%%%%%%%%%%%%%%%%%%%%%%%%%%%%%%%%%%%%%%%%%%%%%%%%%%%%%%%%%
%                                   Einstellungen
%
% Hier können alle relevanten Einstellungen für diese Arbeit gesetzt werden.
% Dazu gehören Angaben u.a. über den Autor sowie Formatierungen.
%
%
%%%%%%%%%%%%%%%%%%%%%%%%%%%%%%%%%%%%%%%%%%%%%%%%%%%%%%%%%%%%%%%%%%%%%%%%%%%%%%%


%%%%%%%%%%%%%%%%%%%%%%%%%%%%%%%%%%%% Sprache %%%%%%%%%%%%%%%%%%%%%%%%%%%%%%%%%%%
%% Aktuell sind Deutsch und Englisch unterstützt.
%% Es werden nicht nur alle vom Dokument erzeugten Texte in
%% der entsprechenden Sprache angezeigt, sondern auch weitere
%% Aspekte angepasst, wie z.B. die Anführungszeichen und
%% Datumsformate.
\setzesprache{en} % de oder en
%%%%%%%%%%%%%%%%%%%%%%%%%%%%%%%%%%%%%%%%%%%%%%%%%%%%%%%%%%%%%%%%%%%%%%%%%%%%%%%%

%%%%%%%%%%%%%%%%%%%%%%%%%%%%%%%%%%% Angaben  %%%%%%%%%%%%%%%%%%%%%%%%%%%%%%%%%%%
%% Die meisten der folgenden Daten werden auf dem
%% Deckblatt angezeigt, einige auch im weiteren Verlauf
%% des Dokuments.



\setzemartrikelnr{802800, 802587}
\setzekurs{MEM3}
\setzejahrgang{2020}
\setzestudienhalbjahr{3}
\setzetitel{Setting up a PLC based Control of a Remote Laboratory Two-Tank System}
\setzedatumAbgabe{30.06.2022}
%\setzefirma{Robert Bosch GmbH}
%\setzeabteilung{eBike - Drive Unit - Engineering Platform (EB-DU-ENP2)}
%\setzefirmenort{Kusterdingen}
\setzeabgabeort{Port Elizabeth}
\setzeabschluss{Master of Science}
\setzestudiengang{Mechatronics}
%\setzedhbw{Reutlingen}
%\setzebetreuer{Egon Konnerth (B. Sc.)}
\setzezeitraum{01.04.2022- 30.06.2022}
\setzemodulbeginn{01.04.2022}
\setzemodulende{30.06.2022}
\setzearbeit{Master-Project}
\setzeautor{Nico Bacher and Jens Krause}
%\setzegutachter{Prof. Dr. Wolfgang Nie\ss{}en}


\inhalttrue                 % auskommentieren oder ändern zu \inhaltfalse, falls kein Inhaltsverzeichnis eingefügt werden soll
\unterschriftenblattfalse    % auskommentieren oder ändern zu \unterschriftenblattfalse, falls kein Unterschriftenblatt eingefügt werden soll
\selbsterkltrue             % auskommentieren oder ändern zu \selbsterklfalse, wenn keine Selbstständigkeitserklärung benötigt wird
\sperrvermerkfalse           % auskommentieren oder ändern zu \sperrvermerkfalse, wenn kein Sperrvermerk benötigt wird
\abkverzfalse                % auskommentieren oder ändern zu \abkverzfalse, wenn kein Abkürzungsverzeichnis benötigt wird
\abbverztrue                % auskommentieren oder ändern zu \abbverzfalse, wenn kein Abbildungsverzeichnis benötigt wird
\tableverzfalse              % auskommentieren oder ändern zu \tableverzfalse, wenn kein Tabellenverzeichnis benötigt wird
\listverzfalse               % auskommentieren oder ändern zu \listverzfalse, wenn kein Listingsverzeichnis benötigt wird
\formelverzfalse             % auskommentieren oder ändern zu \formelverzfalse, wenn kein Formelverzeichnis benötigt wird
\formelgroeverzfalse			% auskommentieren oder ändern zu \formelgroeverzfalse, wenn kein Formelgrößenverzeichnis benötigt wird
\abstractfalse               % auskommentieren oder ändern zu \abstractfalse, wenn kein Abstract gewünscht ist
\bothabstractsfalse          % auskommentieren oder ändern zu \bothabstractsfalse, wenn nur der Abstract in der Hauptsprache eingefügt werden soll
\appendixtrue            % auskommentieren oder ändern zu \appendixfalse, wenn kein Anhang gewünscht ist
\literaturtrue              % auskommentieren oder ändern zu \literaturfalse, wenn kein Literaturverzeichnis gewünscht ist (\appendixtrue muss gesetzt sein!)
\glossarfalse                % auskommentieren oder ändern zu \glossarfalse, wenn kein Glossar gewünscht ist (\appendixtrue muss gesetzt sein!)
\watermarkfalse           % auskommentieren oder ändern zu \watermarktrue, wenn Wasserzeichen auf dem Titelblatt eingefügt werden soll

\refWithPagesfalse          % ändern zu \refWithPagestrue, wenn die Seitenzahl bei Verweisen auf Kapitel engefügt werden sollen


% Angabe des roten/gelben Punktes auf dem Titelblatt zur Kennzeichnung der Vertraulichkeitsstufe.
% Mögliche Angaben sind \yellowdottrue und \reddottrue. Werden beide angegeben, wird der rote Punkt gezeichnet.
% Wird keines der Kommandos angegeben, wird kein Punkt gezeichnet
%\reddottrue
%\yellowdottrue
%\greendottrue

%%%%%%%%%%%%%%%%%%%%%%%%%%%%%%%%%%%%%%%%%%%%%%%%%%%%%%%%%%%%%%%%%%%%%%%%%%%%%%%%

%%%%%%%%%%%%%%%%%%%%%%%%%%%% Literaturverzeichnis %%%%%%%%%%%%%%%%%%%%%%%%%%%%%%
%% Bei Fehlern während der Verarbeitung bitte in ads/header.tex bei der
%% Einbindung des Pakets biblatex (ungefähr ab Zeile 110,
%% einmal für jede Sprache), biber in bibtex ändern.
\newcommand{\ladeliteratur}{%
	\addbibresource{bibliographie.bib}
	%\addbibresource{weitereDatei.bib}
}

%% Zitierstil
%% siehe: http://ctan.mirrorcatalogs.com/macros/latex/contrib/biblatex/doc/biblatex.pdf (3.3.1 Citation Styles)
%% mögliche Werte z.B numeric-comp, alphabetic, authoryear
\setzezitierstil{trad-unsrt}
%%%%%%%%%%%%%%%%%%%%%%%%%%%%%%%%%%%%%%%%%%%%%%%%%%%%%%%%%%%%%%%%%%%%%%%%%%%%%%%%

%%%%%%%%%%%%%%%%%%%%%%%%%%%%%%%%% Layout %%%%%%%%%%%%%%%%%%%%%%%%%%%%%%%%%%%%%%%
%% Verschiedene Schriftarten
% laut nag Warnung: palatino obsolete, use mathpazo, helvet (option scaled=.95), courier instead
\setzeschriftart{palatino} % palatino oder goudysans, lmodern, libertine

%% Abstand vor Kapitelüberschriften zum oberen Seitenrand
\setzekapitelabstand{20pt}

%% Spaltenabstand
\setzespaltenabstand{10pt}
%%Zeilenabstand innerhalb einer Tabelle
\setzezeilenabstand{1.5}
%%%%%%%%%%%%%%%%%%%%%%%%%%%%%%%%%%%%%%%%%%%%%%%%%%%%%%%%%%%%%%%%%%%%%%%%%%%%%%%% % lese Einstellungen

\newcommand{\iflang}[2]{%
	\IfStrEq{\sprache}{#1}{#2}{}
}

\langstr{abkverz}
\langstr{anhang}
\langstr{glossar}
\langstr{deckblattabschlusshinleitung}
\langstr{artikelstudiengang}
\langstr{studiengang}
\langstr{anderdh}
\langstr{von}
\langstr{dbbearbeitungszeit}
\langstr{dbmatriknr}
\langstr{dbkurs}
\langstr{dbfirma}
\langstr{dbbetreuer}
\langstr{dbgutachter}
\langstr{sperrvermerk}
\langstr{erklaerung}
\langstr{abstract}
\langstr{listingname}
\langstr{listlistingname}
\langstr{listingautorefname}
\langstr{selbsterkl}
\langstr{formelsammlung}
\langstr{kopfz}
\langstr{fussz}
\langstr{seite}
\langstr{seitevon}
\langstr{stand}
\langstr{formelgroeverz} % verfügbare Strings
\input{lang/sprache} % Übersetzung einlesen

% Creating own Tab
\newcommand{\myindent}{\hspace*{3cm}}
\newcommand{\myindenti}{\hspace*{0.5cm}}

%\lstset{language=Matlab}
\newcommand{\citem}[1]{\item[\texttt{#1}]} % Code-Item für description-Liste
%\newcommand\todo[1]{\textit{\textcolor{red}{TODO: #1}}\message{LaTeX Warning: \noexpand TODO item left in line \the\inputlineno}} % Todo-Item
\newcommand\todo[1]{\textit{\textcolor{red}{TODO: #1}}} % Todo-Item
\usepackage{pdfpages}         % pdf-Seiten einbinden

%% Farben (Angabe in HTML-Notation mit großen Buchstaben)
\newcommand{\ladefarben}{%
	\definecolor{LinkColor}{HTML}{00007A}
	\definecolor{ListingBackground}{HTML}{FCF7DE}
}
%% Mathematikpakete benutzen (Pakete aktivieren)
%\usepackage{amsmath}
%\usepackage{amssymb}

%% Programmiersprachen Highlighting (Listings)
\newcommand{\listingsettings}{%
	\lstset{%
		language=C++,			% Standardsprache des Quellcodes
		%numbers=left,			% Zeilennummern links
		%stepnumber=1,			% Jede Zeile nummerieren.
		%numbersep=5pt,			% 5pt Abstand zum Quellcode
		%numberstyle=\tiny,		% Zeichengrösse 'tiny' für die Nummern.
		breaklines=true,		% Zeilen umbrechen wenn notwendig.
		breakautoindent=true,	% Nach dem Zeilenumbruch Zeile einrücken.
		postbreak=\space,		% Bei Leerzeichen umbrechen.
		tabsize=2,				% Tabulatorgrösse 2
		basicstyle=\ttfamily\footnotesize, % Nichtproportionale Schrift, klein für den Quellcode
		showspaces=false,		% Leerzeichen nicht anzeigen.
		showstringspaces=false,	% Leerzeichen auch in Strings ('') nicht anzeigen.
		extendedchars=true,		% Alle Zeichen vom Latin1 Zeichensatz anzeigen.
		captionpos=b,			% sets the caption-position to bottom
		%backgroundcolor=\color{ListingBackground}, % Hintergrundfarbe des Quellcodes setzen.
		xleftmargin=0pt,		% Rand links
		xrightmargin=0pt,		% Rand rechts
		frame=single,			% Rahmen an
		frameround=ffff,
		rulecolor=\color{darkgray},	% Rahmenfarbe
		%fillcolor=\color{ListingBackground},
		keywordstyle=\color[rgb]{0.133,0.133,0.6},
		commentstyle=\color[rgb]{0.133,0.545,0.133},
		stringstyle=\color[rgb]{0.627,0.126,0.941},
		aboveskip=1.5em,
	}
}





%%%%%%%%%%%%%%%%%%%%%%%%%%%%% Kopf-/Fußzeilenwechsel %%%%%%%%%%%%%%%%%%%%%%%%%%%
\setlength{\headheight}{40pt}

\newcommand{\setpagestylehead}{%
	\fancypagestyle{plain}{%
		\fancyhf{}
		\fancyhead[L]{\vspace{0.5cm}\footnotesize \nouppercase{\leftmark}}
		\fancyhead[R]{
			\hspace{2.0cm}
			%trim=left bottom right top
			\iflang{en}{
				\begin{textblock*}{188mm}(-50mm,5mm)	\includegraphics[height=2cm]{images/University_Logo.png}
				\end{textblock*}
			}
			\iflang{en}{
				\begin{textblock*}{188mm}(0mm,9mm)
					\includegraphics[height=1.2cm]{images/2000px-HS_Reutlingen_Logo.svg.png}
				\end{textblock*}
			}
			\iflang{de}{
				\begin{textblock*}{188mm}(-50mm,9mm)            	
					\includegraphics[height=1.2cm]{images/2000px-HS_Reutlingen_Logo.svg.png}
				\end{textblock*}
			}
			\iflang{de}{
				%\begin{textblock*}{188mm}(0mm,0mm)            	
				%\includegraphics[height=2.7cm]{images/Bosch_Logo_Kopfzeile}
				%\end{textblock*}
			}
			\iflang{en}{
				%\begin{textblock*}{188mm}(0mm,0mm)            	
				%\includegraphics[height=2.2cm]{images/Bosch_Logo_Kopfzeile_en}
				%\end{textblock*}
			}
		}
		\fancyfoot[L]{
			\noindent{\footnotesize \langfussz\\
				\begin{tabular*}{16cm}{@{\extracolsep{\fill}}l>{\raggedleft}p{8cm}}
					{\footnotesize \langstand: \today} & 
					{\footnotesize \langseite\ \thepage\ \langseitevon\ \pageref*{endOfRomanNumbering}\vspace{1cm}}\tabularnewline
				\end{tabular*}
			}
		}
	}
	\pagestyle{plain}
	\pagenumbering{roman}
}    

\newcommand{\setpagestylecontent}{
	\fancypagestyle{plain}{%
		\fancyhf{}
		\fancyhead[L]{\vspace{0.5cm}\footnotesize \nouppercase{\leftmark}}
		\fancyhead[R]{
			\hspace{2.0cm}
			%trim=left bottom right top
			\iflang{en}{
				\begin{textblock*}{188mm}(-50mm,5mm)	\includegraphics[height=2cm]{images/University_Logo.png}
				\end{textblock*}
			}
			\iflang{en}{
				\begin{textblock*}{188mm}(0mm,9mm)	\includegraphics[height=1.2cm]{images/2000px-HS_Reutlingen_Logo.svg.png}
				\end{textblock*}
			}
			\iflang{de}{
				\begin{textblock*}{188mm}(0mm,9mm)            	
					\includegraphics[height=1.2cm]{images/2000px-HS_Reutlingen_Logo.svg.png}
				\end{textblock*}
			}
			\iflang{de}{
				%\begin{textblock*}{188mm}(0mm,0mm)            	
				%\includegraphics[height=2.7cm]{images/Bosch_Logo_Kopfzeile}
				%\end{textblock*}
			}
			\iflang{en}{
				%\begin{textblock*}{188mm}(0mm,0mm)            	
				%\includegraphics[height=2.2cm]{images/Bosch_Logo_Kopfzeile_en}
				%\end{textblock*}
			}
		}
		\fancyfoot[L]{
			\noindent{\footnotesize \langfussz\\
				\begin{tabular*}{16cm}{@{\extracolsep{\fill}}l>{\raggedleft}p{8cm}}
					{\footnotesize \langstand: \today} & 
					{\footnotesize \langseite\ \thepage\ \langseitevon\ \pageref*{endOfArabicNumbering}\vspace{1cm}}\tabularnewline
				\end{tabular*}
			}
		}
	}
	\pagestyle{plain}
	\pagenumbering{arabic}
}

\newcommand{\setpagestylefoot}{
	\fancypagestyle{plain}{%
		\fancyhf{}
		\fancyhead[L]{\vspace{0.5cm}\footnotesize \nouppercase{\leftmark}}
		\fancyhead[R]{
			\hspace{2.0cm}
			%trim=left bottom right top
			\iflang{de}{
				\begin{textblock*}{188mm}(0mm,9mm)            	
					\includegraphics[height=1.4cm]{images/2000px-HS_Reutlingen_Logo.svg.png}
				\end{textblock*}
			}
			\iflang{en}{
				\begin{textblock*}{188mm}(0mm,9mm)            	
					\includegraphics[height=1.2cm]{images/2000px-HS_Reutlingen_Logo.svg.png}
				\end{textblock*}
			}
			\iflang{de}{
				%\begin{textblock*}{188mm}(0mm,0mm)            	
				%\includegraphics[height=2.7cm]{images/Bosch_Logo_Kopfzeile}
				%\end{textblock*}
			}
			\iflang{en}{
				%\begin{textblock*}{188mm}(0mm,0mm)            	
				%\includegraphics[height=2.2cm]{images/Bosch_Logo_Kopfzeile_en}
				%\end{textblock*}
			}
		}
		\fancyfoot[L]{
			\noindent{\footnotesize \langfussz\\
				\begin{tabular*}{16cm}{@{\extracolsep{\fill}}l>{\raggedleft}p{8cm}}
					{\footnotesize \langstand: \today} & 
					{\footnotesize \langseite\ \thepage\ \langseitevon\ \pageref*{LastPage}\vspace{1cm}}\tabularnewline
				\end{tabular*}
			}
		}
	}
	\pagestyle{plain}
	\pagenumbering{Alph}
}


%%%%%%%%%%%%%%%%%%%%%%%%%%%%%%%%%%%%%%%%%%%%%%%%%%%%%%%%%%%%%%%%%%%%%%%%%%%%%%%%

% Einstellung der Sprache des Paketes Babel und der Verzeichnisüberschriften

\iflang{de}{
	\usepackage[english, ngerman]{babel}
	\selectlanguage{ngerman}
}
\iflang{en}{
	\usepackage[ngerman, english]{babel}
	\selectlanguage{english}
}

\usepackage[utf8]{inputenc}
\usepackage[T1]{fontenc}
\usepackage{tikz}
\usepackage{xcolor}
\usepackage{additionalPackages/tikz-uml} % UML Diagramme
\usepackage[european]{additionalPackages/circuitikz}


%%%%%%% Package Includes %%%%%%%

\usepackage[margin=2.5cm,foot=1cm,top=3cm,bottom=3cm]{geometry}	% Seitenränder und Abstände
\usepackage[activate]{microtype} %Zeilenumbruch und mehr
\usepackage[onehalfspacing]{setspace}
\usepackage{makeidx}
\usepackage[autostyle=true,german=quotes]{csquotes}
\usepackage{longtable}
\usepackage{enumitem}	% mehr Optionen bei Aufzählungen
\usepackage{graphicx}
\usepackage{xcolor} 	% für HTML-Notation
\usepackage{float}
\usepackage{array}
\usepackage{calc}		% zum Rechnen (Bildtabelle in Deckblatt)
\usepackage[right]{eurosym}
\usepackage{wrapfig}
\usepackage{pgffor} % für automatische Kapiteldateieinbindung
\usepackage[perpage, hang, multiple, stable]{footmisc} % Fussnoten
\usepackage{acronym}
\usepackage[absolute]{textpos}
%\usepackage[printonlyused, footnote]{acronym} % falls gewünscht kann die Option footnote eingefügt werden, dann wird die Erklärung nicht inline sondern in einer Fußnote dargestellt
\usepackage{scrhack} % in Kombination mit listings-Package kommt es zu Warnings, dieses Paket verhindert die Warnings! Ggf. auskommentieren und die Warnings akzeptieren falls Verzeichnisse nicht so dargestellt werden wie gewünscht
\usepackage{listings} % Code-Listings
%\usepackage[numbered, framed]{matlab-prettifier}
\usepackage[framed]{matlab-prettifier} % .sty-Datei muss vorhanden sein! Kann auskommentiert werden, falls keine Matlab-Listings in der Arbeit enthalten sind.
\usepackage{color, colortbl}  %Für Highlighten der Tabellenzeilen
\usepackage{amsmath}% http://ctan.org/pkg/amsmath
\usepackage{tabu}
\usepackage{tabularx}
\usepackage{multirow}


% eine Kommentarumgebung "k" (Handhabe mit \begin{k}<Kommentartext>\end{k},
% Kommentare werden rot gedruckt). Wird \% vor excludecomment{k} entfernt,
% werden keine Kommentare mehr gedruckt.
\usepackage{comment}
\specialcomment{k}{\begingroup\color{red}}{\endgroup}
%\excludecomment{k}


%%%%%% Configuration %%%%%

%% Anwenden der Einstellungen

\usepackage{\schriftart}
\ladefarben{}

% Titel, Autor und Datum
\title{\titel}
\author{\autor}
\date{\datum}

%\usepackage[list=true]{sub}

% PDF Einstellungen
\usepackage[%
pdftitle={\titel},
pdfauthor={\autor},
pdfsubject={\arbeit},
pdfcreator={pdflatex, LaTeX with KOMA-Script},
pdfpagemode=UseOutlines, 		% Beim Oeffnen Inhaltsverzeichnis anzeigen
pdfdisplaydoctitle=true, 		% Dokumenttitel statt Dateiname anzeigen.
pdflang={\sprache}, 			% Sprache des Dokuments.
]{hyperref}

% (Farb-)einstellungen für die Links im PDF
\hypersetup{%
	colorlinks=true, 		% Aktivieren von farbigen Links im Dokument
	linkcolor=black, 	    % Farbe festlegen
	citecolor=LinkColor,
	filecolor=LinkColor,
	menucolor=LinkColor,
	urlcolor=LinkColor,
	%linktocpage=true, 		% Nicht der Text sondern die Seitenzahlen in Verzeichnissen klickbar
	linktoc=all,            % Seitenzahlen und Text klickbar
	bookmarksnumbered=true 	% Überschriftsnummerierung im PDF Inhalt anzeigen.
}
% Workaround um Fehler in Hyperref, muss hier stehen bleiben
\usepackage{bookmark} %nur ein latex-Durchlauf für die Aktualisierung von Verzeichnissen nötig

% Caption linksbündig
\usepackage{caption}
% \captionsetup{justification   = raggedright, singlelinecheck = false}
\captionsetup{belowskip=-5pt}
% Schriftart in Captions etwas kleiner
\addtokomafont{caption}{\small}

\usepackage{subfig}

% Literaturverweise (sowohl deutsch als auch englisch)
%\iflang{de}{%
%	\usepackage[
%	backend=bibtex,		% empfohlen. Falls biber Probleme macht: bibtex
%	bibwarn=true,
%	bibencoding=utf8,	% wenn .bib in utf8, sonst ascii
%	sortlocale=de_DE,
%	style=\zitierstil,
%	]{biblatex}
%}
%\iflang{en}{%
%	\usepackage[
%	backend=bibtex,		% empfohlen. Falls biber Probleme macht: bibtex
%	bibwarn=true,
%	bibencoding=utf8,	% wenn .bib in utf8, sonst ascii
%	sortlocale=en_US,
%	style=\zitierstil,
%	sortcites=true,
%	]{biblatex}
%}


%\ladeliteratur{}
%\bibliography{bibliographie.bib}

% Glossar
\usepackage[nonumberlist,toc]{glossaries}
\usepackage{blindtext} % Blindtext-Package. Common Usage: \blindtext für einzelnen Abschnitt, \Blindtext für mehrere Abschnitte

%%%%%% Additional settings %%%%%%

% Hurenkinder und Schusterjungen verhindern
% http://projekte.dante.de/DanteFAQ/Silbentrennung
\clubpenalty = 10000 % schließt Schusterjungen aus (Seitenumbruch nach der ersten Zeile eines neuen Absatzes)
\widowpenalty = 10000 % schließt Hurenkinder aus (die letzte Zeile eines Absatzes steht auf einer neuen Seite)
\displaywidowpenalty=10000

\setcounter{biburlnumpenalty}{100}
\setcounter{biburlucpenalty}{100}
\setcounter{biburllcpenalty}{100}

% Bildpfad
\graphicspath{{images/}}

% Einige häufig verwendete Sprachen
\lstloadlanguages{PHP,Python,Java,C,C++,bash}
\listingsettings{}
% Umbennung des Listings
\renewcommand\lstlistingname{\langlistingname}
\renewcommand\lstlistlistingname{\langlistlistingname}
\def\lstlistingautorefname{\langlistingautorefname}

% Abstände in Tabellen
\setlength{\tabcolsep}{\spaltenabstand}
\renewcommand{\arraystretch}{\zeilenabstand}

\usepackage{xspace}
\newcommand{\lastcontentpage}{}
\usepackage{amsfonts}

\usetikzlibrary{shapes,arrows,calc}
\usepackage{relsize}

\usepackage{censor}

\usepackage{eso-pic}


%% Paket um Textteile drehen zu können
%\usepackage{rotating}
%% Paket um Seite im Querformat anzuzeigen
%\usepackage{lscape}

\newcommand\Watermark{%
	\put(0,0){%
		\parbox[b][\paperheight]{\paperwidth}{%
			\vfill
			\includepdf[scale=0.8,angle=50,pages={1},pagecommand={}]{ads/watermark}
			\vfill
		}
	}
}

\ifrefWithPages
%RJG8FE: add a pageref to autoref whenever the referenced page is not the same as the current one
%        useful for printed documents without clickable hyperlinks
\AtBeginDocument{\let\oldautoref\autoref}
\AtBeginDocument{
	\renewcommand{\autoref}[1]{%
		\oldautoref{#1}%
		\ifthenelse{\thepage=\pageref{#1}}% if current page number equals the referenced page number
		{}% then add nothing
		{ (S. \pageref{#1})}% else add the text
	}
}
\fi

\usepackage{amssymb} % Erweiterung der Symbole in Mathematikumgebung

\iflang{de}{\usepackage{icomma}} % Europäisches Komma in Formeln

\usepackage{xr}
\usepackage{makecell}	%break line in tables
\usepackage{longtable}
\usepackage{units}
\usepackage{gensymb}