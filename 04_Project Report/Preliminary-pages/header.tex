%!TEX root = ../IPEB3 Project Report.tex

\documentclass[12pt, parskip=half]{article} 
%=========================Preamble===================================
\usepackage[utf8]{inputenc}
\usepackage[T1]{fontenc} %font encoding
\usepackage{mathptmx}    %adobe times roman font
\usepackage{xstring}
\usepackage{lastpage}
\usepackage{fancyhdr}
\usepackage{refcount}
\usepackage{amsmath,amssymb,amsfonts} %ams packages for math
\usepackage{setspace} %for line spacing
\usepackage[top=2cm,bottom=2.6cm,left=2.6cm,right=2.6cm]{geometry} %for margins
\usepackage{graphicx}\graphicspath{{Graphics/}} %to add images
\usepackage{subcaption} %for subfig.
\usepackage{framed}
\usepackage{tocloft}
\usepackage{emptypage}
\usepackage{booktabs,multirow,bigstrut,colortbl}
\usepackage{enumerate}
\usepackage{tabularx}
\renewcommand{\arraystretch}{1}
\usepackage{tabulary}
\usepackage[activate]{microtype}
\usepackage{threeparttable}
\usepackage{dblfloatfix}
\usepackage[sort&compress,numbers]{natbib} %format bibliography
\bibliographystyle{vancouver}
\renewcommand{\bibname}{References}
\usepackage{titlesec,xcolor}
\usepackage{adjustbox}
\usepackage{tikz}
\usepackage[most]{tcolorbox}
\usepackage{wrapfig}
\usepackage{pgfplots}
\pgfplotsset{compat=newest}
\usepackage[intoc,refpage]{nomencl}\makenomenclature %for list of symbols
\renewcommand{\nomname}{List of Symbols}
\renewcommand{\pagedeclaration}[1]{\dotfill\hyperpage{#1}}

\usepackage[nottoc]{tocbibind} %adds ref and ind in toc
\usepackage[toc,page]{appendix} %creats a new page for appendices title
\usepackage{varioref}
\usepackage[hidelinks]{hyperref} %creats the links
\usepackage[noabbrev]{cleveref}
\usepackage{fancyhdr}%\pagestyle{fancy} %for fancy headers and footers
%\fancyhf{}
\renewcommand{\headrulewidth}{0pt}
\usepackage{tocloft}
\usepackage{algorithm}% http://ctan.org/pkg/algorithm
%\PassOptionsToPackage{noend}{algpseudocode}% comment out if want end's to show
\usepackage{algpseudocode}% http://ctan.org/pkg/algorithmicx

\renewcommand{\appendixtocname}{Appendices}
\renewcommand{\appendixpagename}{\vspace*{\fill}\centering Appendices\vspace*{\fill}}

%           Defining colors for sticky notes:
%_______________________________________________________________
% Yellow:
\definecolor{BgYellow}{HTML}{FFF59C}
\definecolor{FrameYellow}{HTML}{F7A600}

% Green:
\definecolor{BgGreen}{HTML}{C7D92D}
\definecolor{FrameGreen}{HTML}{89B23B}

% Blue:
\definecolor{BgBlue}{HTML}{45BEE9}
\definecolor{FrameBlue}{HTML}{31A8C9}

% Defining Sticky note box:
\newtcolorbox{StickyNote}[1][]{%
	enhanced,
	before skip=2mm,after skip=2mm, 
	width=1\textwidth, boxrule=0.2mm, % width of the sticky note
	colback=BgYellow, colframe=FrameYellow, % Colors
	attach boxed title to top left={xshift=0cm,yshift*=0mm-\tcboxedtitleheight},
	varwidth boxed title*=-1cm,
	% The titlebox:
	boxed title style={frame code={%
			\path[left color=FrameYellow,right color=FrameYellow,
			middle color=FrameYellow]
			([xshift=-0mm]frame.north west) -- ([xshift=0mm]frame.north east)
			[rounded corners=0mm]-- ([xshift=0mm,yshift=0mm]frame.north east)
			-- (frame.south east) -- (frame.south west)
			-- ([xshift=0mm,yshift=0mm]frame.north west)
			[sharp corners]-- cycle;
		},interior engine=empty,
	},
	sharp corners,rounded corners=southeast,arc is angular,arc=3mm,
	% The "folded paper" in the bottom right corner:
	underlay={%
		\path[fill=BgYellow!80!black] ([yshift=3mm]interior.south east)--++(-0.4,-0.1)--++(0.1,-0.2);
		\path[draw=FrameYellow,shorten <=-0.05mm,shorten >=-0.05mm,color=FrameYellow] ([yshift=3mm]interior.south east)--++(-0.4,-0.1)--++(0.1,-0.2);
	},
	drop fuzzy shadow, % Shadow
	fonttitle=\bfseries, 
	title={#1}
}

% Blue Sticky Note (BStkyNote):
\newtcolorbox{BStkyNote}[1][]{%
	enhanced,
	before skip=2mm,after skip=2mm, 
	width=1\textwidth, % width of the sticky note
	boxrule=0.2mm,
	colback=BgBlue, colframe=FrameBlue, % Colors
	attach boxed title to top left={xshift=0cm,yshift*=0mm-\tcboxedtitleheight},
	varwidth boxed title*=-3cm,
	% The titlebox:
	boxed title style={frame code={%
			\path[left color=FrameBlue,right color=FrameBlue,
			middle color=FrameBlue]
			([xshift=-0mm]frame.north west) -- ([xshift=0mm]frame.north east)
			[rounded corners=0mm]-- ([xshift=0mm,yshift=0mm]frame.north east)
			-- (frame.south east) -- (frame.south west)
			-- ([xshift=0mm,yshift=0mm]frame.north west)
			[sharp corners]-- cycle;
		},interior engine=empty,
	},
	sharp corners,rounded corners=southeast,arc is angular,arc=3mm,
	% The "folded paper" in the bottom right corner:
	underlay={%
		\path[fill=BgBlue!80!black] ([yshift=3mm]interior.south east)--++(-0.4,-0.1)--++(0.1,-0.2);
		\path[draw=FrameBlue,shorten <=-0.05mm,shorten >=-0.05mm,color=FrameBlue] ([yshift=3mm]interior.south east)--++(-0.4,-0.1)--++(0.1,-0.2);
	},
	drop fuzzy shadow, % Shadow
	fonttitle=\bfseries, 
	title={#1}
}

% Green Sticky Note (GStkyNote):
\newtcolorbox{GStkyNote}[1][]{%
	enhanced,
	before skip=2mm,after skip=2mm, 
	width=1\textwidth, % width of the sticky note
	boxrule=0.2mm,
	colback=BgGreen, colframe=FrameGreen, % Colors
	attach boxed title to top left={xshift=0cm,yshift*=0mm-\tcboxedtitleheight},
	varwidth boxed title*=-3cm,
	% The titlebox:
	boxed title style={frame code={%
			\path[left color=FrameGreen,right color=FrameGreen,
			middle color=FrameGreen]
			([xshift=-0mm]frame.north west) -- ([xshift=0mm]frame.north east)
			[rounded corners=0mm]-- ([xshift=0mm,yshift=0mm]frame.north east)
			-- (frame.south east) -- (frame.south west)
			-- ([xshift=0mm,yshift=0mm]frame.north west)
			[sharp corners]-- cycle;
		},interior engine=empty,
	},
	sharp corners,rounded corners=southeast,arc is angular,arc=3mm,
	% The "folded paper" in the bottom right corner:
	underlay={%
		\path[fill=BgGreen!80!black] ([yshift=3mm]interior.south east)--++(-0.4,-0.1)--++(0.1,-0.2);
		\path[draw=FrameGreen,shorten <=-0.05mm,shorten >=-0.05mm,color=FrameGreen] ([yshift=3mm]interior.south east)--++(-0.4,-0.1)--++(0.1,-0.2);
	},
	drop fuzzy shadow, % Shadow
	fonttitle=\bfseries, 
	title={#1}
}
\usepackage{listings}
\renewcommand{\lstlistingname}{Code}
\renewcommand{\lstlistlistingname}{List of Code Sections}

%\lstloadlanguages{C,C++,csh,Java}

\definecolor{red}{rgb}{0.6,0,0} 
\definecolor{blue}{rgb}{0,0,0.6}
\definecolor{green}{rgb}{0,0.4,0}
\definecolor{cyan}{rgb}{0.0,0.6,0.6}
\definecolor{cloudwhite}{rgb}{0.9412, 0.9608, 0.8471} 
\definecolor{gainsboro}{rgb}{0.99, 0.99, 0.99}
\definecolor{rose}{cmyk}{0,0.26,0.38,0}
\lstset{
	language=csh,
	basicstyle=\footnotesize\ttfamily,
	numbers=left,
	numberstyle=\tiny,
	numbersep=5pt,
	tabsize=2,
	extendedchars=true,
	breaklines=true,
	frame=b,
	stringstyle=\color{rose},
	showspaces=false,
	showtabs=false,
	xleftmargin=17pt,
	framexleftmargin=17pt,
	framexrightmargin=5pt,
	framexbottommargin=4pt,
	commentstyle=\color{green},
	morecomment=[l]{//}, %use comment-line-style!
	morecomment=[s]{/*}{*/}, %for multiline comments
	showstringspaces=false,
	morekeywords={ abstract, event, new, struct,
		as, explicit, null, switch, def,
		base, extern, object, this,
		bool, false, operator, throw,
		break, finally, out, true,
		byte, fixed, override, try,
		case, float, params, typeof,
		catch, for, private, uint,
		char, foreach, protected, ulong,
		checked, goto, public, unchecked,
		class, if, readonly, unsafe,
		const, implicit, ref, ushort,
		continue, in, return, using,
		decimal, int, sbyte, virtual,
		default, interface, sealed, volatile,
		delegate, internal, short, void,
		do, is, sizeof, while,
		double, lock, stackalloc,
		else, long, static,
		enum, namespace, string},
	keywordstyle=\color{blue},
	identifierstyle=\color{cyan},
	backgroundcolor=\color{gainsboro},
}

\DeclareCaptionFont{black}{\color{black}}
\DeclareCaptionFormat{listing}{\colorbox{gray!12}{\parbox{\textwidth}{\hspace{15pt}#1#2#3}}}
\captionsetup[lstlisting]{format=listing,labelfont=black,textfont=black, singlelinecheck=true, margin=0pt, font={bf,footnotesize}}

\makeatletter
\let\ps@plain\ps@fancy
\makeatother


\setcounter{secnumdepth}{3}
\setcounter{tocdepth}{3}
\renewcommand{\contentsname}{Table of Contents}

%New Commands
\newcommand{\namesigdate}[2][1.5cm]{%	
	\begin{tabular}{@{}p{#1}@{}}
		#2 \\\fromsig{\includegraphics[scale=0.3]{signature.png}} \hrule
		{\small \textit{signature}}
		
	\end{tabular}
}
%------------
\raggedbottom	