%================ch2======================================
\chapter{Literature}\label{ch:ch2}
Literature, most generically, is any body of written works. More restrictively, literature refers to writing considered to be an art form or any single writing deemed to have artistic or intellectual value, often due to deploying language in ways that differ from ordinary usage.

\section{Equation}
One of the greatest motivating forces for Donald Knuth when he began developing the original TeX system was to create something that allowed simple construction of mathematical formulae, while looking professional when printed. The fact that he succeeded was most probably why TeX (and later on, LaTeX) became so popular within the scientific community. Typesetting mathematics is one of LaTeX's greatest strengths. It is also a large topic due to the existence of so much mathematical notation. 

The velocity, v ($v=d/t$) is ....
\nomenclature{$v$}{velocity, m/s}
\nomenclature{$d$}{distance, m}
\nomenclature{$t$}{time, s}
%%
\begin{equation}
\nu = \frac{\mu}{\rho}
\end{equation}
\nomenclature{$\nu$}{momentum diffusivity, $\mathrm{m^2/s}$}

\begin{table}
\centering
\caption{Random table-2}
\label{tab:sample2}
\begin{tabular}{|c|c|c|c|c|c|c|c|c|c|}
\hline 
a & b & c & d & b & c & d & b & c & d \\ \hline 
5 & 6 & 7 & 8 & b & c & d & b & c & d\\ \hline 
9 & 10 & 11 & 12 & b & c & d & b & c & d \\ \hline 
a & b & c & d & b & c & d & b & c & d\\ \hline 
5 & 6 & 7 & 8 & b & c & d & b & c & d\\ \hline 
9 & 10 & 11 & 12 & b & c & d & b & c & d \\ \hline 
a & b & c & d & b & c & d & b & c & d\\ \hline 
\end{tabular} 
\end{table}