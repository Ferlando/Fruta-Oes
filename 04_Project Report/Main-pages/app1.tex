%%================app1======================================
\vspace{-15mm}
\section{Library importing and Device Setup}\label{app:app1}
\vspace{-7mm}
\begin{lstlisting}[language={python}, caption={Library importing and Device Setup}, label=Initialization]
	#!/usr/bin/env python3
	import ev3dev.ev3 as ev3
	from ev3dev.ev3 import * #to allow the use of brick buttons
	import math #importing Math Library to allow usage of Mathematical functions
	import ColorSensor2 #importing Color sensor library
	from time import sleep #importing time library
	import time
	#-----------------------------------------------------------------
	#Device Connections start
	# Connect the outputs to the motor
	motor_left = ev3.LargeMotor('outA')     
	motor_right = ev3.LargeMotor('outD')
	gripper_motor = ev3.MediumMotor('outC')
	
	#Gyro Sensor initialisation
	gy = ev3.GyroSensor()
	
	#Colour Sensor initialisation
	cl= ColorSensor2.ColorSensor2()
	cl.mode='COL-COLOR'
	
	#Ultra-Sonic Sensor initialisation
	us = ev3.UltrasonicSensor() 
	us.mode='US-DIST-CM'
	
	btn = Button() #activate the use of brick buttons
	
	#Global Variables (we know use of global variables is not a good practice, but in this case it is necessary)
	harvested = 0  #keeping count of the harvested fruits
	total_distance = 0  #keeping track of the distance travelled by the robot
	gripper_opened = True   #keeping track of the state of the gripper
	thetha = None   #To avoid the scanning angle from exceeding the limits
	start = 0   #Variable to start the 5 minute timer
	end = 0 #Variable to end the 5 minute timer
	current_time = 0    #Variable to track the current time
	#Device Connections ends
\end{lstlisting}


\section{User Menu for Selecting the Colour}
\vspace{-7mm}
\begin{lstlisting}[language={python}, caption={User Menu for Selecting the Colour}, label=menu]
def menu():
#Function to select the desired fruit using the buttons from the Brick
global start    
ev3.Sound.speak("Select the colour from the buttons after the beep").wait() 
sleep(3)
Sound.beep().wait()
start = time.time() #Starting the timer
while True:
if btn.any():# Checks if any button is pressed.
button_id = btn.buttons_pressed
print(button_id[0])
break
else:
sleep(0.01)

option_text = " I will search for {0}" #Robot confirming the colour
ev3.Sound.speak(option_text.format(fruit_dict.get(button_id[0]))).wait() 
return fruit_dict.get(button_id[0]) #Returning the desired fruit
\end{lstlisting}

\section{Function for turning left}
\vspace{-7mm}
\begin{lstlisting}[language={python}, caption={Function for turning left}, label=leftTurn]
def turn_left(speed,angle):
#Function that turns left, it requires the turning speed and the angle it must turn
gy.mode = 'GYRO-RATE'
gy.mode = 'GYRO-ANG'
sleep(1)
while gy.value()>-angle:
motor_left.run_forever(speed_sp=-speed, stop_action = 'hold')
motor_right.run_forever(speed_sp=speed, stop_action = 'hold')
#print(gy.value())
motor_left.stop()
motor_right.stop()
\end{lstlisting}

\section{Function for turning right}
\vspace{-7mm}
\begin{lstlisting}[language={python}, caption={Function for turning right}, label=rightTurn]
def turn_right(speed,angle):
#Function that turns right, it requires the turning speed and the angle it must turn
gy.mode = 'GYRO-RATE'
gy.mode = 'GYRO-ANG'
sleep(1)
while gy.value()<angle:
motor_left.run_forever(speed_sp=speed, stop_action = 'hold')
motor_right.run_forever(speed_sp=-speed, stop_action = 'hold')
#print(gy.value())
motor_left.stop()
motor_right.stop()
\end{lstlisting}

\section{Function for moving the robot forward}
\vspace{-7mm}
\begin{lstlisting}[language={python}, caption={Function for moving the robot forward}, label=moveDistance]
def move_distance_in_mm(distance_mm):
#Function to move a specified distance, it transforms rotation of the wheels to distance.
dist = (distance_mm/1000) * (360/0.176)
motor_left.run_to_rel_pos(position_sp = dist, speed_sp=400, stop_action = "brake")
motor_right.run_to_rel_pos(position_sp = dist, speed_sp=400, stop_action = "brake")
motor_left.wait_while('running')
motor_right.wait_while('running')
motor_left.stop()
motor_right.stop()
\end{lstlisting}
\newpage
\section{Function for moving the robot from Acre to Home-Zone}
\vspace{-7mm}
\begin{lstlisting}[language={python}, caption={Function for moving the robot from Acre to Home-Zone}, label=driveToHomeZone]
def drive_to_home_zone(hypotenuse, current_angle, add):
#Function to drive to the home zone from the Acre with a correct fruit.
global total_distance, harvested, start, end, current_time
distance = abs(hypotenuse*(math.sin(math.radians(current_angle)))) #Using Pythagorean theorem to calculate the distance from the fruit to the home-zone
total_distance += abs(hypotenuse*(math.cos(math.radians(current_angle)))) #Adding the distance moved forward to the total for returning to the starting point
turn_right(100,current_angle + 85) #turning right 90 degrees (we used 85 because the Gyro-sensor was not accurate)
gy.mode = 'GYRO-RATE'
gy.mode = 'GYRO-ANG'
sleep(1)
move_distance_in_mm(abs(distance + add)) 
gripper("open") #Open the gripper to release the fruit in the home-zone
harvested += 1  #keeping count of the harvested fruits
move_distance_in_mm(-add) #returning to the main-line
turn_left(200,85) #Turning left to continue searching
end = time.time()
current_time += (end - start) #Checking the current time
\end{lstlisting}

\section{Function for moving the robot from Acre to Main-Line}
\vspace{-7mm}
\begin{lstlisting}[language={python}, caption={Function for moving the robot from Acre to Main-Line}, label=driveFromAcreToLine]
def drive_from_acre_to_line(hypotenuse, current_angle):
#Function to drive to the main-line from the Acre without any fruit.
global total_distance, start, end, current_time
distance = abs(hypotenuse*(math.sin(math.radians(current_angle))))
print(distance)
turn_left(1000, 80) #moving a wrong fruit away so that it cannot be detected again.
sleep(1)
turn_right(200,current_angle) #turning the robot perpendicular to the main line after moving a wrong fruit.
total_distance += abs(hypotenuse*(math.cos(math.radians(current_angle)))) #Adding the distance moved forward to the total for returning to the starting point
gy.mode = 'GYRO-RATE'
gy.mode = 'GYRO-ANG'
sleep(1)
move_distance_in_mm(-abs(distance)) #Reversing the robot to the main-line
\end{lstlisting}

\section{Function for operating the Gripper}
\vspace{-7mm}
\begin{lstlisting}[language={python}, caption={Function for operating the Gripper}, label=gripper]
def gripper(command):
#Function to open and close the gripper using the Medium Motor
global gripper_opened
if (command == "close" and gripper_opened == True):
gripper_motor.run_to_rel_pos(speed_sp=150, position_sp=230, stop_action = 'brake') # closing
sleep(2)
gripper_opened = False
elif (command == "open" and gripper_opened == False):
gripper_motor.run_to_rel_pos(speed_sp=250, position_sp=-210, stop_action = 'brake') #opening
gripper_motor.wait_while('running')
gripper_opened = True
\end{lstlisting}

\section{Function for Searching the fruits}
\vspace{-7mm}
\begin{lstlisting}[language={python}, caption={Function for Searching the fruits}, label=scanning]
def scanning():
#Function for scanning the acre for nearby fruits
global total_distance, start, end, current_time, thetha
gy.mode = 'GYRO-RATE'
gy.mode = 'GYRO-ANG'
sleep(1)
search_space = 1500 #scanning for fruits withing 1.5m (Width of the acre)
scanning_speed = 50 
dist = us.value() #Storing the distance given by the Ultra-Sonic sensor
while(dist > search_space or thetha == None): #Only detect fruits which are withing the search space
while(dist > search_space and gy.value()>-85): #It scans turning left 90 degrees
motor_left.run_forever(speed_sp=-scanning_speed, stop_action = 'hold')
motor_right.run_forever(speed_sp=scanning_speed, stop_action = 'hold')
dist = us.value() #Distance at which the fruit is detected
motor_left.stop()
motor_right.stop()

while(dist > search_space and gy.value()<-5): #it scans turning right 90 degrees
motor_left.run_forever(speed_sp=scanning_speed, stop_action = 'hold')
motor_right.run_forever(speed_sp=-scanning_speed, stop_action = 'hold')
dist = us.value() #Distance at which the fruit is detected
motor_left.stop()
motor_right.stop()

thetha = gy.value() #Angle at which the fruit is detected
dist =us.value() #Distance at which the fruit is detected

if(dist > search_space): # If no fruits detected, advance the robot 20 cm
move_distance_in_mm(200)
total_distance += 200 #Adding the distance moved forward to the total for returning to the starting point

end = time.time()
current_time += (end - start) #Checking the current time

while(us.value() < search_space): #optimizing the exact angle at which the fruit is detected, turn the robot until the fruit is lost/no_longer_detected
motor_left.run_forever(speed_sp=-50, stop_action = 'hold')
motor_right.run_forever(speed_sp=50, stop_action = 'hold')    
motor_left.stop()
motor_right.stop()
alpha = gy.value() - thetha #Angle at which the fruit is lost when turning

while (gy.value()< (thetha + (alpha/2))): #Aligning the robot with the center of the fruit
motor_left.run_forever(speed_sp=50, stop_action = 'hold')
motor_right.run_forever(speed_sp=-50, stop_action = 'hold')
motor_left.stop()
motor_right.stop()

if(dist > 500): #If the fruit is more than 0.5m away, the robot must stop halfway and scan again.
move_distance_in_mm(dist/2)
while(us.value() < search_space):
motor_left.run_forever(speed_sp=-50, stop_action = 'hold')
motor_right.run_forever(speed_sp=50, stop_action = 'hold')    
motor_left.stop()
motor_right.stop()
alpha = gy.value() - thetha

while (gy.value()< (thetha + (alpha/2))):
motor_left.run_forever(speed_sp=50, stop_action = 'hold')
motor_right.run_forever(speed_sp=-50, stop_action = 'hold')
motor_left.stop()
motor_right.stop()
move_distance_in_mm(dist/2+10)
else:
move_distance_in_mm(dist+10) #Moving forward 1cm more to force the fruit to be below the colour sensor

sleep(1)
gripper('close') #Closing the gripper to allow colour sensing function

if (cl.getCalibratedColorString() == fruit): #if the sensored fruit is correct
ev3.Sound.speak("Right")
drive_to_home_zone(dist, abs(gy.value()), 300) #Drive it to the Home-zone
else:
ev3.Sound.speak("wrong") #if the sensored fruit is wrong
gripper('open') #open the gripper
drive_from_acre_to_line(dist, abs(gy.value())) #move the fruit away   
turn_right(200,85)
end = time.time()
current_time += (end - start) #Checking the current time
\end{lstlisting}

\section{Main Code}
\vspace{-7mm}
\begin{lstlisting}[language={python}, caption={Main Code}, label=main]
fruit_dict ={   #Declaring the available fruits
	#Dictionary of all possible fruits (Coloured cubes) in the competition and corresponding selection button on the EV3 brick
	'up':"Red", 
	"right":"Green",
	"left":"Blue",
	"down":"Yellow",
}

fruit = menu() #Storing the desired fruit

while (harvested<2  and current_time<300): #when it reaches 300s (5 minutes) or when it harvested 2 fruits, it must come back to the starting point
scanning()
if(total_distance >= 4000): #If the robot drove the whole length of the acre, it must return to the starting point and scan again
move_distance_in_mm(-total_distance)
total_distance = 0

move_distance_in_mm(-total_distance) #Reversing the robot back to the starting Position
\end{lstlisting}